\documentclass[OPS,lsstdraft,authoryear,toc]{lsstdoc}
\input{meta}

% Package imports go here.

% Local commands go here.

%If you want glossaries
%\input{aglossary.tex}
%\makeglossaries

\title{Data Preview 1: Definition and planning}

% This can write metadata into the PDF.
% Update keywords and author information as necessary.
\hypersetup{
    pdftitle={Data Preview 1: Definition and planning},
    pdfauthor={William O'Mullane},
    pdfkeywords={}
}

% Optional subtitle
% \setDocSubtitle{A subtitle}

\author{%
William O'Mullane
}

\setDocRef{RTN-085}
\setDocUpstreamLocation{\url{https://github.com/lsst/rtn-085}}

\date{\vcsDate}

% Optional: name of the document's curator
% \setDocCurator{The Curator of this Document}

\setDocAbstract{%
This document is to centralise information and planing for DP1. It will point to other documents for details.
}

% Change history defined here.
% Order: oldest first.
% Fields: VERSION, DATE, DESCRIPTION, OWNER NAME.
% See LPM-51 for version number policy.
\setDocChangeRecord{%
  \addtohist{1}{YYYY-MM-DD}{Unreleased.}{William O'Mullane}
}


\begin{document}

% Create the title page.
\maketitle
% Frequently for a technote we do not want a title page  uncomment this to remove the title page and changelog.
% use \mkshorttitle to remove the extra pages

\section{Introduction} \label{sec:intro}

We pull together information from Jira and refer to other documents as needed to describe DP1 in more detail.

DP1 should be released to the community after the First Look image is released to the press.
The general order with some time (days potentially weeks)  between steps is:

\begin{itemize}
\item Public First Look - JPG and large images full focal plane or more
\item DP1 - ComCam data reprocessed and in good form
\item Science Images in the FL images
\end{itemize}


\subsection{DP1 Scope} \label{sec:scope}
The goal od DP1 is to give data to the science community - not any new features in RSP etc.
\subsubsection{Data}
DP1 comprises data from ComCam on sky.
\citeds{RTN-011} gives an outline of DP1 however at time of writing it still states DP1 could be either camera.

DP1 could contain data with poor PSF as ComCam may be driven by work on AOS verification.
As such:
\begin{itemize}
\item There will be PVIs and catalogs if the data are good enough.
\item There is no promise of coadds but if there is good data this is a stretch goal.
\item There will be no difference imaging in DP1.
\end{itemize}

\paragraph{First Look Images} \label{sec:fl}

First Look images from LSSTCam will not be part of DP! but should follow sometime after it.
The FL will only be what as done for producing the images of which there should be a small number.
It should have PVIs PSF matched coadds  and  may have catalogs if it is feasible.
DM currently have no visibility on what kind of imaging will be taken so can not commit to anything for FL.

\subsubsection{Science Platform}
No new functionality is expected on RSP for DP1 - no full focal place viewer is required.
The image cutout service is to be more robust and scalable but bulk cutouts will not be supported for DP1.


Catalogs will be served by Qserv at USDF as baseline.






\section{Planning} \label{sec:plan}

We have some milestones and epics leading up to DP1.

We are missing some like Qserv - loading ..

Dates are wrong ..

Label milestones oand epics DP1 to have them appear

\begin{figure}
\begin{centering}
\includegraphics[width=0.9\textwidth]{DP1}
	\caption{Data Preview 1 - plan \label{fig:plan}}
\end{centering}
\end{figure}



Open milestones are listed in \tabref{tab:openMilestones}.
\input{openMilestones}


\subsection{Component plans}
\subsubsection{Middleware}
We need client server butler \jira{PREOPS-3697}

Any Workflow needs ?

\subsection{RSP}

\paragraph{Qserv}
Data loading.


\subsection{Rucio}
What do we need for data tp be sent to DACs?




\appendix
% Include all the relevant bib files.
% https://lsst-texmf.lsst.io/lsstdoc.html#bibliographies
\section{References} \label{sec:bib}
\renewcommand{\refname}{} % Suppress default Bibliography section
\bibliography{local,lsst,lsst-dm,refs_ads,refs,books}

% Make sure lsst-texmf/bin/generateAcronyms.py is in your path
\section{Acronyms} \label{sec:acronyms}
\addtocounter{table}{-1}
\begin{longtable}{p{0.145\textwidth}p{0.8\textwidth}}\hline
\textbf{Acronym} & \textbf{Description}  \\\hline

AOS & Active Optics System \\\hline
AP & Alert Production \\\hline
CCD & Charge-Coupled Device \\\hline
DEC & Declination \\\hline
DM & Data Management \\\hline
DMLT & DM Leadership Team \\\hline
DMS & Data Management Subsystem \\\hline
DMS-REQ & Data Management System Requirements prefix \\\hline
DP & Data Production \\\hline
DP0 & Data Preview 0 \\\hline
DP1 & Data Preview 1 \\\hline
FL & First Look \\\hline
FY24 & Financial Year 24 \\\hline
FY25 & Financial Year 25 \\\hline
HIPS & Hierarchical Progressive Survey (IVOA standard) \\\hline
IDAC & Independent Data Access Center \\\hline
JPG & Joint Photographic Experts Group \\\hline
L1 & Lens 1 \\\hline
L2 & Lens 2 \\\hline
L3 & Lens 3 \\\hline
LVV & LSST Verification and Validation \\\hline
OPS & Operations \\\hline
PSF & Point Spread Function \\\hline
PVI & Processed Visit Image \\\hline
QA & Quality Assurance \\\hline
RA & Risk Assessment \\\hline
RDM & Rubin Data Management \\\hline
RPF & Rubin system PerFormance \\\hline
RSP & Rubin Science Platform \\\hline
RTN & Rubin Technical Note \\\hline
SIA & Simple Image Access (IVOA standard) \\\hline
US & United States \\\hline
\end{longtable}

% If you want glossary uncomment below -- comment out the two lines above
%\printglossaries





\end{document}
