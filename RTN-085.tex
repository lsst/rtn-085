\documentclass[OPS,authoryear,toc]{lsstdoc}
\input{meta}

% Package imports go here.

% Local commands go here.

%If you want glossaries
%\input{aglossary.tex}
%\makeglossaries

\title{Data Preview 1: Definition and planning}

% This can write metadata into the PDF.
% Update keywords and author information as necessary.
\hypersetup{
    pdftitle={Data Preview 1: Definition and planning},
    pdfauthor={William O'Mullane},
    pdfkeywords={}
}

% Optional subtitle
% \setDocSubtitle{A subtitle}

\author{%
William O'Mullane
}

\setDocRef{RTN-085}
\setDocUpstreamLocation{\url{https://github.com/lsst/rtn-085}}

\date{\vcsDate}

% Optional: name of the document's curator
% \setDocCurator{The Curator of this Document}

\setDocAbstract{%
This document is to centralise information and planing for DP1. It will point to other documents for details.
}

% Change history defined here.
% Order: oldest first.
% Fields: VERSION, DATE, DESCRIPTION, OWNER NAME.
% See LPM-51 for version number policy.
\setDocChangeRecord{%
  \addtohist{1}{YYYY-MM-DD}{Unreleased.}{William O'Mullane}
}


\begin{document}

% Create the title page.
\maketitle
% Frequently for a technote we do not want a title page  uncomment this to remove the title page and changelog.
% use \mkshorttitle to remove the extra pages

% ADD CONTENT HERE
% You can also use the \input command to include several content files.

\appendix
% Include all the relevant bib files.
% https://lsst-texmf.lsst.io/lsstdoc.html#bibliographies
\section{References} \label{sec:bib}
\renewcommand{\refname}{} % Suppress default Bibliography section
\bibliography{local,lsst,lsst-dm,refs_ads,refs,books}

% Make sure lsst-texmf/bin/generateAcronyms.py is in your path
\section{Acronyms} \label{sec:acronyms}
\addtocounter{table}{-1}
\begin{longtable}{p{0.145\textwidth}p{0.8\textwidth}}\hline
\textbf{Acronym} & \textbf{Description}  \\\hline

AOS & Active Optics System \\\hline
AP & Alert Production \\\hline
CCD & Charge-Coupled Device \\\hline
DEC & Declination \\\hline
DM & Data Management \\\hline
DMLT & DM Leadership Team \\\hline
DMS & Data Management Subsystem \\\hline
DMS-REQ & Data Management System Requirements prefix \\\hline
DP & Data Production \\\hline
DP0 & Data Preview 0 \\\hline
DP1 & Data Preview 1 \\\hline
FL & First Look \\\hline
FY24 & Financial Year 24 \\\hline
FY25 & Financial Year 25 \\\hline
HIPS & Hierarchical Progressive Survey (IVOA standard) \\\hline
IDAC & Independent Data Access Center \\\hline
JPG & Joint Photographic Experts Group \\\hline
L1 & Lens 1 \\\hline
L2 & Lens 2 \\\hline
L3 & Lens 3 \\\hline
LVV & LSST Verification and Validation \\\hline
OPS & Operations \\\hline
PSF & Point Spread Function \\\hline
PVI & Processed Visit Image \\\hline
QA & Quality Assurance \\\hline
RA & Risk Assessment \\\hline
RDM & Rubin Data Management \\\hline
RPF & Rubin system PerFormance \\\hline
RSP & Rubin Science Platform \\\hline
RTN & Rubin Technical Note \\\hline
SIA & Simple Image Access (IVOA standard) \\\hline
US & United States \\\hline
\end{longtable}

% If you want glossary uncomment below -- comment out the two lines above
%\printglossaries





\end{document}
